\documentclass[14pt, a4paper]{extarticle}

% --- Packages ---

% Experimental
% \usepackage[none]{hyphenat}

\usepackage{polyglossia} % Localization
\usepackage{fontspec}
\usepackage[T1]{fontenc}
\usepackage{pdfpages} % Importing PDF files
\usepackage{tocvsec2} % Controll over sections in the table of contents
\usepackage{longtable} % Support for tables continuing to the next page
\usepackage{booktabs} % The package enhances the quality of tables in LATEX, providing extra commands as well as behind-the-scenes optimisation
\usepackage{array} % An extended implementation of the array and tabular environments which extends the options for column formats, and provides “programmable” format specifications
\usepackage{setspace} % Provides support for setting the spacing between lines in a document
\usepackage{graphicx} % Required for inserting images
\usepackage[newfloat]{minted} % Code highliting
\usepackage[labelsep=period,justification=centering]{caption} % Captions
\usepackage{amsmath} % Math symbol support
\usepackage{url} % Support of url links
\usepackage{multirow} % Tables with merged rows
\usepackage{hyperref} % Links integration
\usepackage{indentfirst} % Paragraph indentation
\usepackage{chngcntr} % Captions and enumeration of images
\usepackage{ragged2e} % Justification of text
\usepackage{titlesec} % Section management
\usepackage[
    style=gost-footnote,
    backend=biber,
    % bibencoding=utf8
    sorting=nyt,
    language=russian,
    autolang=other,
    citestyle=gost-numeric,
    bibstyle=gost-numeric,
    opcittracker=false
]{biblatex} % Bibliography management
\usepackage{csquotes} % quotes for other languages
\usepackage{fancyhdr} % For something...
\usepackage{float} % For table placements

\usepackage[left=3cm,right=1.5cm,top=2cm,bottom=2cm]{geometry} % Padding

% --- Formatting ---

% Language support
\setdefaultlanguage[spelling=modern]{russian}
\setotherlanguage{english}
% Font settings
\setmainfont{Times New Roman} 
\setmonofont{Times New Roman}
\setromanfont{Times New Roman} 
\newfontfamily\cyrillicfont[
    Ligatures=TeX,    
]{Times New Roman}
\newfontfamily\cyrillicfonttt{JetBrainsMono-Regular.otf}[
  Script=Cyrillic,
  Path=fonts/,
]
% Set mono font
\setmonofont{JetBrainsMono-Regular.otf}[
    SizeFeatures={Size=10},
    Path = fonts/,
    Ligatures=TeX,
    Contextuals=Alternate
]
% Fix page numbers
\fancyhf{}
\fancyfoot[C]{\cyrillicfont\thepage}
\renewcommand{\headrulewidth}{0pt}
\pagestyle{fancy}
% Other
\urlstyle{same} % Url font style
\linespread{1.5} % Gap between lines should be 1.5
\setlength{\parindent}{1.25cm} % Paragraph gap = 1.25 cm
\graphicspath{ {./images/} } % Path to the images directory
\hypersetup{
    colorlinks=true, 
    linkcolor=black,
    filecolor=blue, 
    citecolor = black,       
    urlcolor=blue,
} % Links and citations coloring

% Make figures, tables, listings be enumerated with sections
% \counterwithin{figure}{section}
% \counterwithin{table}{section}
% \counterwithin{listing}{section}

% Configure listings
\SetupFloatingEnvironment{listing}{name={Код}}
\newenvironment{code}{
    \captionsetup[listing]{hypcap=false}
    \vbox\bgroup
}{
    \smallskip
    \egroup
}
\newenvironment{codeappendix}{
    \captionsetup[listing]{hypcap=false}
}{
    \smallskip
}
\setminted{
    frame=lines,
    framesep=2mm,
    baselinestretch=1.2,
    fontsize=\footnotesize
}

\SetupFloatingEnvironment{table}{name={Таблица}}
\captionsetup[table]{
    justification=raggedright,
    singlelinecheck=false
}
% \newenvironment{table}{
%     \captionsetup[table]{hypcap=false}
%     \vbox\bgroup
% }{
%     \smallskip
%     \egroup
% }

% --- Macros ---

% Make square brackets for cases
\makeatletter
\newenvironment{sqcases}{%
  \matrix@check\sqcases\env@sqcases
}{%
  \endarray\right.%
}
\def\env@sqcases{%
  \let\@ifnextchar\new@ifnextchar
  \left\lbrack
  \def\arraystretch{1.2}%
  \array{@{}l@{\quad}l@{}}%
}
\makeatother

\newcommand{\image}[3]{
\begin{figure}[H]
	\centering
	\includegraphics[width=#3\textwidth]{#1}
	\caption{#2}
\end{figure}
}

% \renewenvironment{appendix}{
%     \captionsetup[appendix]{hypcap=false}
%     \vbox\bgroup
% }{
%     \smallskip
%     \egroup
% }

\newcommand{\codefromfile}[2]{
\begin{codeappendix}
    \inputminted[breaklines=true, framesep=2mm, fontsize=\footnotesize, firstline=1, linenos]{#2}{listings/#1}
\end{codeappendix}
}
% \newcommand{\codefromfile}[3]{
% \begin{codeappendix}
%     \inputminted[breaklines=true, framesep=2mm, fontsize=\footnotesize, firstline=1,]{#2}{listings/#1}
%     \captionof{listing}{#3}
% \end{codeappendix}
% }

% --- Other ---

% Add bibliography reference
\addbibresource{bibliography.bib}
% Disable hyphenation
\tolerance=1
\emergencystretch=\maxdimen
\hyphenpenalty=10000
\hbadness=10000
% Disable section and subsection enumeration
\setcounter{secnumdepth}{0}
% Make sections and subsections centered
\titleformat{\section}[block]{\Large\bfseries\filcenter}{}{1em}{}
\titleformat{\subsection}[block]{\large\bfseries\filcenter}{}{1em}{}

\begin{document}

% --- Title page ---

% \includepdf[pages=-]{titlepage.pdf}
\setcounter{page}{2}

% --- Table of contents ---

\newpage 
\begin{center}
    \tableofcontents
\end{center}
\newpage 

% --- Contents ---

\justifying

\phantomsection
\section{Теоритическая часть}


\newpage
\phantomsection
\section{Практическая часть}


% --- Bibliography ---

\newpage
\phantomsection
\nocite{*}
\printbibliography
\addcontentsline{toc}{section}{Список литературы}

% --- Appendix ---

\newpage
\phantomsection
\section{Приложение}

\RaggedLeft

% \phantomsection
% \subsection*{Приложение 1. Код решения задачи 1}
% \label{appendix:1}
% \codefromfile{work_1.R}{R}
%
% \newpage
% \phantomsection
% \subsection*{Приложение 2. Код решения задачи 2}
% \label{appendix:2}
% \codefromfile{work_2.R}{R}
%
% \newpage
% \phantomsection
% \subsection*{Приложение 3. Код решения задачи 3}
% \label{appendix:3}
% \codefromfile{work_3.R}{R}
%
% \newpage
% \phantomsection
% \subsection*{Приложение 4. Код решения задачи 4}
% \label{appendix:4}
% \codefromfile{work_4.py}{python}
%
% \newpage
% \phantomsection
% \subsection*{Приложение 5. Код решения задачи 5}
% \label{appendix:5}
% \codefromfile{work_5.py}{python}
%
%
% \newpage
% \phantomsection
% Приложение 6
% \image{images/plot_2.png}{Схема усеченного решающего дерева, полученного при решении задачи 4.}{1}
% \label{appendix:6}
%
% \newpage
% \phantomsection
% Приложение 7
% \begin{code}
% \begin{minted}{python}
% numerical_features = ['first_occurrence_date', 'last_occurrence_date',
%                       'reported_date', 'geo_x', 'geo_y', 'geo_lat', 'geo_lon',
%                       'victim_count', 'reported_time', 'reported_time_utc']
% numerical_data = pd.DataFrame(index=data.index, columns=numerical_features)
% numerical_data.loc[:, numerical_features] = data.loc[:,
%     data.columns.isin(numerical_features)]
%
% time_features = ['first_occurrence_date',
%                  'last_occurrence_date', 'reported_date']
%
% for col in time_features:
%     numerical_data.loc[:, col] = numerical_data[col].apply(
%         lambda timestr: pd.to_datetime(timestr, format='%m/%d/%Y %I:%M:%S %p'))
% numerical_data['reported_time'] = \
%     numerical_data['reported_date'].apply(lambda x: pd.to_datetime(
%         x.strftime('%H:%M:%S')))
% numerical_data.loc[:, 'reported_time_utc'] = \
%     numerical_data['reported_time'].dt.hour * 60 * 60 + \
%     numerical_data['reported_time'].dt.minute * 60 + \
%     numerical_data['reported_time'].dt.second
% numerical_data = numerical_data.astype('float64', errors='ignore')
% # Отбросим значения даты и времени в формате Timestamp для нормализации
% normalizible_features = ['geo_x', 'geo_y', 'geo_lat',
%                          'geo_lon', 'victim_count', 'reported_time_utc']
% normalizible_data = numerical_data.loc[:, numerical_data.columns.isin(
%     normalizible_features)]
% \end{minted}
% \captionof{listing}{Нормализация числовых данных, представленных в датасете.}
% \end{code}
% \label{appendix:7}
%
% \newpage
% \phantomsection
% Приложение 8
% \begin{codeappendix}
% \begin{minted}{python}
% geo_x                1.000000
% geo_y                0.988746
% geo_lat             -0.989791
% geo_lon              0.999873
% victim_count         0.000309
% reported_time_utc   -0.004854
% Name: geo_x, dtype: float64 
% geo_x                0.988746
% geo_y                1.000000
% geo_lat             -0.962731
% geo_lon              0.988911
% victim_count         0.001351
% reported_time_utc   -0.003204
% Name: geo_y, dtype: float64 
% geo_x               -0.989791
% geo_y               -0.962731
% geo_lat              1.000000
% geo_lon             -0.987968
% victim_count         0.000482
% reported_time_utc    0.005366
% Name: geo_lat, dtype: float64 
% geo_x                0.999873
% geo_y                0.988911
% geo_lat             -0.987968
% geo_lon              1.000000
% victim_count         0.000331
% reported_time_utc   -0.004929
% Name: geo_lon, dtype: float64 
% geo_x                0.000309
% geo_y                0.001351
% geo_lat              0.000482
% geo_lon              0.000331
% victim_count         1.000000
% reported_time_utc   -0.003938
% Name: victim_count, dtype: float64 
% geo_x               -0.004854
% geo_y               -0.003204
% geo_lat              0.005366
% geo_lon             -0.004929
% victim_count        -0.003938
% reported_time_utc    1.000000
% Name: reported_time_utc, dtype: float64 
% \end{minted}
% \captionof{listing}{Вывод \hyperref[code:41]{кода 41} (Нахождение коррелирующих признаков, представленных в датасете).}
% \end{codeappendix}
% \label{appendix:8}
%
% \newpage
% \phantomsection
% Приложение 9
% \begin{codeappendix}
% \begin{minted}{python}
% clean_features = ['offence_code+extension', 'reported_time_utc',
%                   'geo_x', 'geo_y', 'district_id', 'is_crime',
%                   'is_traffic', 'victim_count']
% clean_data = pd.DataFrame(columns=clean_features, index=data.index)
%
% clean_data['offence_code+extension'] = data['offense_code'] * \
%     10 + data['offense_code_extension']
% clean_data['reported_time_utc'] = numerical_data['reported_time_utc']
% clean_data['geo_x'] = numerical_data['geo_x']
% clean_data['geo_y'] = numerical_data['geo_y']
% clean_data['district_id'] = data['district_id']
% clean_data['is_crime'] = data['is_crime']
% clean_data['is_traffic'] = data['is_traffic']
% clean_data['victim_count'] = data['victim_count']
% clean_data = clean_data.dropna()
% # print(f'Number of objects in cleaned dataset: {len(clean_data.index)}')
%
% normalizible_clean_features = ['reported_time_utc', 'geo_x', 'geo_y',
%                                'victim_count']
% normalizible_clean_data = clean_data.loc[:, clean_data.columns.isin(
%     normalizible_clean_features)]
% std_scaler = preprocessing.StandardScaler()
% normalized_clean_data = pd.DataFrame(
%     std_scaler.fit_transform(normalizible_clean_data),
%     columns=normalizible_clean_data.columns,
%     index=normalizible_clean_data.index)
%
% for col in normalizible_clean_features:
%     clean_data[col] = normalized_clean_data.loc[:, col]
% clean_data = clean_data.apply(
%     lambda x: pd.to_numeric(x, errors='coerce')).dropna()
%
% clean_data.to_csv('./clean_crime.csv')
% \end{minted}
% \captionof{listing}{Подготовка данных к применению метода PCA.}
% \end{codeappendix}
% \label{appendix:9}
%
% \newpage
% \phantomsection
% Приложение 10
% \image{images/plot_5.png}{График доли дисперсии объясненной каждой из компонент после применения метода PCA.}{1}
% \label{appendix:10}
%
% \newpage
% \phantomsection
% Приложение 11
% \image{images/plot_8.png}{Двумерная визуализация данных, полученная после применения алгоритма t-SNE (окраска по критерию \textit{victim\_count}).}{1}
% \label{appendix:11}
%
% \newpage
% \phantomsection
% Приложение 12
% \image{images/plot_9.png}{Двумерная визуализация данных, полученная после применения алгоритма t-SNE (окраска по критерию \textit{offence\_code+extension}).}{1}
% \label{appendix:12}


\end{document}
